\documentclass[11pt]{article}

% Language setting
% Replace `english' with e.g. `spanish' to change the document language
\usepackage[english]{babel}

% Set page size and margins
% Replace `letterpaper' with `a4paper' for UK/EU standard size
\usepackage[letterpaper,top=2cm,bottom=2cm,left=3cm,right=3cm,marginparwidth=1.75cm]{geometry}
\usepackage[utf8]{inputenc}
\usepackage{cite}
% Useful packages
\usepackage{amsmath}
\usepackage{graphicx}
\usepackage[colorlinks=true, allcolors=blue]{hyperref}

\title{\textbf{\Huge Exploring the Potential Shifts in Our Understanding of Space and Time through Quantum Gravity}}
\author{Riddhiman Bhattacharya}

\begin{document}
\maketitle
\Large
\begin{abstract}
In this paper, I tried to provide an overview of how various quantum gravity approaches prompt us to reconsider our understanding of space and time. The primary focus has been on two prominent contenders: string theory and loop quantum gravity. However, it's important to bear in mind that these theories remain unverified and lack a universally accepted consensus.

As we navigate through these ideas, it becomes evident that our conventional notions of space and time might necessitate a fundamental shift. The very fabric of spacetime could reveal intricacies that challenge our prior assumptions. Moreover, our exploration will encompass diverse viewpoints on the nature of time within the realm of quantum gravity.
\end{abstract}



\section*{Introduction}
This brief review paper offers an introduction to potential transformations in our understanding of space and time prompted by a theory of quantum gravity~\cite{butterfield2011aemergence,dieks2015emergence,barbour2000end}.

Given the inherent limitations of this overview, it presents a selective viewpoint~\cite{henson2012causal}. It caters to philosophers without an extensive physics background, though it assumes a certain level of familiarity with fundamental philosophical concepts.

It is crucial to acknowledge that there exists no universally embraced and empirically validated theory of quantum gravity~\cite{polchinski1998string}. The notions outlined herein are tentative, rooted in reputable modern approaches to quantum gravity and their implications for the fundamental nature of space and time \cite{becker2006string}.

The pursuit of unraveling the intrinsic essence of space and time constitutes a metaphysical enterprise~\cite{rickles2013dual,rickles2011philosopher,rickles2008quantum}.In modern philosophy, trying to understand the very essence of reality is part of metaphysics. When we talk about the fundamental nature of things, it's a philosophical topic. Metaphysical ideas are also woven into scientific theories when we think about them in a realistic way. Some philosophers say that metaphysics should be its own separate field, but many believe that a deep exploration of metaphysics should consider what we've learned from science.

It's important to be cautious when using the word "metaphysics" with physicists. Some might not react well to it because it can have negative meanings from the past, when logical positivism was influential. Most physicists nowadays see science in a more realistic light and have their own philosophical ideas, even if they don't explicitly call it "metaphysics."~\cite{deharoforthcoming,Teh2016-TEHGAG-2}.



\section{Requirement of Quantum Gravity Theory}

In the 20th century, our view of the physical world changed a lot because of new ideas from relativity \cite{belot2011background} and quantum physics \cite{zettili2009quantum,greene1999elegant,thorne1995black}. Relativity says space and time are one thing, called \textbf{spacetime }\cite{carroll2004spacetime}. Quantum physics is even more different, but it hasn't changed space and time yet. To make a new idea for quantum gravity, we need to use quantum rules for space-time. This might change how we think about space and time. Quantum gravity things matter most at the very tiny Planck size. We use the speed of light ($c$), the gravity number ($G$), and Planck's number ($h$) to find lengths, times, and energy. For example, Planck's size is about $10^{-35}$ meters \cite{callender2001physics}. At this size, we need to think about quantum gravity, and our old space-time idea might not work \cite{rovelli2004quantum}.

People are still talking about what quantum physics means for reality. There are also questions about relativity. Even though these ideas are hard, we know they are really true and help us in many ways \cite{hawking1988brief}.

In science today, there are four big forces: \textbf{gravity}, \textbf{electromagnetism}, \textbf{weak}, and \textbf{strong}. Einstein's gravity idea is good, but others use quantum ideas. The problem is, gravity and the other forces don't fit together. We need a new way to mix quantum and gravity ideas, like for black holes or the start of the universe \cite{smolin2006trouble,thorne1995black,hawking1996nature}.

Making a quantum gravity idea is hard. We don't have lots of proof to help us, and testing is tricky \cite{cox2012quantum}. So, we start with the ideas we already know and try to make them work with both quantum and gravity. Making a new idea is tough, but it can't go against what we know from quantum and relativity \cite{zettili2009quantum}. We've different ways we're thinking about it, but they must fit with what we know. We hope to find a good idea that we can test with proof someday \cite{hawking1996nature}.





\section{Potential Shifts in Our Understanding of Space and Time through Quantum Gravity}

How might a theory of quantum gravity change the way we think about space and time? To tackle this question, we've to explore the different approaches to quantum gravity that have been developed. These approaches suggest various ways to adjust our understanding of space and time, although they share some similarities. Regardless of the specific approach, a successful theory must explain how our everyday experience of space and time emerges, and it should align with the achievements of our current theories. Essentially, any theory of quantum gravity should be able to at least somewhat mirror the space-time concepts we're familiar with from General Relativity (GR). However, it remains uncertain how precisely a more fundamental theory would approximate GR.

\subsection{Fundamentality of Space-time}

A common idea that many quantum gravity approaches share is that space-time could be considered as "emergent" in some way \cite{butterfield2011aemergence}. However, this notion of emergence in the context of quantum gravity isn't as strong as it might be in other philosophical discussions. When we say spacetime is emergent here, we mean it's not a fundamental part of the core theory of how the world works; rather, it can be derived from this more fundamental theory. This concept of emergence is meant to align with the idea of reductionism \cite{butterfield2011bemergence}.

One prevalent concept suggested by various approaches is that we might need to change our perspective on space and time by redefining how we describe space-time's nature \cite{Huggett_2013,huggett2015deriving,witten1996reflections}. Instead of viewing spacetime as a continuous entity, it might be more accurate to describe it as a discrete structure. One such approach that explores this idea is causal set theory (CST). Although CST isn't as developed as loop quantum gravity or string theory, it offers a unique perspective. However, a complete quantum version of CST hasn't been fully formulated yet\cite{wuthrich2012structure}.

In CST \cite{wuthrich2012structure}, the starting point is a collection of discrete elements that form a partially ordered set based on a fundamental binary relation. The fundamental properties of space-time are thought to arise from this underlying network of discrete elements. Certain causal set networks can closely approximate the descriptions of spacetime that we're familiar with from General Relativity (GR) \cite{belot2011background}. In GR, a spacetime's description encodes which points in spacetime can be causally connected; this means signals or objects can be transmitted from earlier events to later ones. When a causal set approximates a GR spacetime, the binary relation used to establish the partial order can be interpreted as a causal relation that signifies which events can influence others in a causal manner.
Another discrete perspective emerges in loop quantum gravity (LQG) \cite{gambini2011first,kiefer2012quantum,rickles2006structural} . In LQG, discreteness isn't initially assumed but arises as a consequence of other underlying principles. Quantum states in this theory can be visually represented by something called a "spin network." It's important to note that this network, or more accurately, a quantum superposition of such networks, doesn't reside within space; rather, it forms the very fabric of space itself. Within the framework of LQG, areas and volumes become quantized and take on specific, discrete values. However, not all solutions of LQG lead to familiar spacetime structures. One challenge in LQG is to find solutions that give rise to conventional space-time descriptions. For instance, it hasn't yet been demonstrated that LQG can generate solutions corresponding to Minkowski space-time, which represents the fundamental flat spacetime \cite{oriti2009approaches}.

Among physicists, the most widely discussed approach to quantum gravity is string theory \cite{zwiebach2009first}. At first glance, string theory appears to offer a relatively conventional view of spacetime, albeit with additional dimensions beyond the familiar four-dimensional spacetime. In fact, super-symmetric string theories require as many as 10 dimensions.

Critics of string theory \cite{smolin2006trouble} often raise concerns about its "background dependence," contending that the theory relies on a fixed background spacetime being assumed from the outset \cite{dawid2006underdetermination}. In response, proponents of string theory argue that it is actually background independent, asserting that the geometry of the background must satisfy Einstein's field equations. Polchinski (1998) \cite{polchinski1998string} provides a detailed derivation of this result. The conflicting viewpoints arise from differing criteria for determining true background dependence \cite{becker2006string}.

Nevertheless, it's reasonable to acknowledge that string theory exhibits a form of background independence, albeit in a weaker and less explicit manner compared to the more overt background independence observed in LQG.

At its core, string theory proposes a shift in our understanding of fundamental particles. Rather than viewing particles as point-like entities, string theory suggests they are extended one-dimensional structures known as strings. As these strings traverse space-time, they trace out two-dimensional surfaces called world-sheets. In the realm of quantum theory, this concept appears to enable the unification of all fundamental forces, including gravity, into a single coherent framework. Different types of particles correspond to various vibrational states of these quantum strings \cite{wuthrich2005to}. Among these states, one can be identified as the graviton—a quantum particle responsible for conveying gravitational interactions. Notably, the properties of particles are influenced by the background within which the strings move. An analogy often used likens different particles to distinct musical tones produced by the string, while the background serves as a resonance box dictating which tones can be emitted. A flat background leads to either massless particles or particles with masses far beyond those encountered in the standard model of particle physics. Traditionally, background manifolds have been conceived as representations of spacetime—a perspective that has raised concerns. These concerns are amplified by the concept of "dualities," wherein two seemingly different descriptions are actually physically equivalent. In string theory, dual descriptions may involve background "spacetimes" with vastly distinct geometries or topologies. Although physicists assert that dual pictures convey the same underlying reality, this undermines a direct interpretation of background manifolds as accurate portrayals of spacetime \cite{wuthrich2010no}.

Huggett \cite{huggettforthcoming} introduces a compelling argument against equating background manifolds with spacetime. This argument draws from the concept of T-dualities in string theory, where two distinct background manifolds form a dual pair, featuring circular compact dimensions with differing radii. T-duality reveals that the larger of the two radii corresponds to the radius of the effective or perceived spacetime. This insight underscores the notion that background manifolds cannot be directly identified as representations of spacetime. Moreover, this principle extends to more intricate dualities, even encompassing cases where the topology between the dual pair of backgrounds diverges.

While string theory initially appears to offer a more conventional treatment of space-time, even with the inclusion of extra dimensions, the presence of dualities adds complexity to this perspective. To gain insight, we can emphasize that the two-dimensional world-sheets traced by strings are fundamentally more significant than the higher-dimensional space-time they inhabit. In string theory's quantum framework, the behavior of strings is described through quantum fields on these world-sheets. The intrinsic properties of space-time are thought to emerge from the collective quantum behavior of numerous strings.

In practice, practical calculations often require the introduction of a classical background. Determining the effective or observable space-time from this background can be challenging. Notably, the emergence of space-time in string theory is not primarily about the contrast between continuous and discrete structures.

Another duality within string theory supporting the notion of space-time emergence is the \textbf{Anti-de Sitter / Conformal Field Theory correspondence (AdS/CFT)} \cite{rickles2013ads}. While this duality lacks rigorous proof, it is widely believed to hold true. In this scenario, a four-dimensional quantum field theory corresponds to a 10-dimensional string theory. Some scholars, such as de Mello Koch \cite{demellokoch2012emergent} and Murugan (2012)\cite{murugan2012foundations}, have suggested that this implies the emergence of six spacetime dimensions. However, philosophers of physics have raised questions about whether one theory genuinely emerges from the other, particularly if the dual theories are physically equivalent, as discussed by Rickles (2013) \cite{rickles2013dual} and Teh (2013)\cite{teh2013holography,Teh2016-TEHGAG-2}.

Dieks, van Dongen, and de Haro (2015) \cite{dieks2015emergence} as well as de Haro \cite{deharoforthcoming} have proposed that emergence from one picture to the other could be feasible if the duality is only approximate. Given that the duality lacks strict proof, this possibility remains open for exploration.

\subsection{Time in Quantum Gravity}

When we try to come up with a theory that combines quantum physics and gravity, we run into a tricky issue known as the "problem of time." This challenge is especially prominent in certain ways of studying gravity, like \textbf{Loop Quantum Gravity}. The trouble arises when we try to use a method called "canonical formalism," \cite{rickles2006time} where we split space-time into pieces and introduce a time element to understand how space changes. In everyday situations, this approach works well and gives us a space-time picture. However, in General Relativity (GR), which is about gravity, there's no fixed time that everyone agrees on. This means we can start with different time references, and still, end up with the same space-time.

Now, when we want to apply quantum ideas to gravity using this method, something odd happens. The flexibility in choosing a time reference seems to make time almost irrelevant. This leads to a strange situation where things that should change over time don't change at all. This goes against what we know from our everyday experience. This "problem of time" has led to different ways of thinking.

Some experts suggest that when we build a theory that combines quantum physics and gravity, we might need to treat space and time differently. Scientist Barbour \cite{barbour2000end} proposes a radical idea – to get rid of time from our basic understanding of reality. According to him, reality is made up of separate moments in three-dimensional space, sort of like individual snapshots. These moments aren't connected like a continuous space-time, and time isn't a fundamental part of this setup. Instead, the appearance of time comes from these snapshots, but time itself isn't part of the main picture.

On the other hand, Prof. Smolin has a different perspective. He suggests that time should be the fundamental thing, even if space isn't as important \cite{smolin2013time}. He believes that only the present moment exists, and it changes because of a real and important time. This view is called "presentism." It requires a special way of dividing space-time, but this division can't be measured precisely. Still, Smolin argues that there's an objective way to decide which events are happening at the same time and which ones happened earlier\cite{}.

These ideas from Barbour and Smolin are quite different from each other and also from the usual way we think about space and time in theories like General Relativity. It's interesting to note that they both challenge the idea of "relativity of simultaneity," which is an essential part of our current understanding. They offer new ways of thinking about space and time in the context of quantum gravity.

There are other viewpoints too. Rovelli suggests that we shouldn't give up on the idea of the relativity of simultaneity \cite{rovelli2004quantum,rovelli2007quantum}. He thinks that there might not be a single objective time, and this is a lesson we learned from theories like General Relativity. Rovelli's view is that both space and time are not fundamental but emerge in a similar way. He emphasizes that when we measure how things change, we're actually comparing different things, not using a fixed external time \cite{matsubara2013realism}.

All these different ideas show that the problem of time has triggered various responses. It's important to explore these ideas as we work towards a theory of quantum gravity, even though we don't have a definitive answer yet. This diversity of perspectives can help us make progress in understanding the complex relationship between quantum physics and gravity.

\section{Conclusion}

In this brief review, I've aimed to provide an overview of how modern research in quantum gravity is suggesting new and revolutionary ideas about space and time.

Some thinkers have pondered whether solving the mystery of how quantum physics relates to measurement and interpretation could also shed light on the challenge of creating a quantum theory of gravity. While this connection isn't confirmed, it's worth considering. It's possible that different viewpoints or interpretations might fit more naturally with different approaches to quantum gravity.

Another interesting idea to consider is that different approaches could come together in some way. Maybe each approach has discovered unique pieces of the puzzle, and the way forward could involve blending ideas from different sources in a creative manner.

\newpage
\bibliographystyle{unsrt}
\bibliography{sample}


\end{document}